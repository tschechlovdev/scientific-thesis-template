\chapter{Evaluation of Hyperparameter Selection}

\section{Offline}

\subsection{Setup}

\begin{itemize}
    \item For the evaluation, 81 datasets that were all synthetically generated were used.
    \item For the generation of the dataset the tool described in \cite{Fritz2019InitializingAnalysis} was used. \xtodo{Maybe move this part to implementation? And then only describe the used input parameters. Or maybe describe also training and test data generation with used parameters in implementation}
    \item The generated the datasets based on the following input parameters:
    \begin{itemize}
        \item $n$: Number of instances of the dataset.
        \item $d$: Number of dimensions (or features) of the dataset.
        The value of each generated feature was in the intervall [-10, 10].
        \item $k$: Number of clusters of the dataset. Each cluster was generated according to the Gaussian distribution with mean at the center and standard deviation of 0.5.
        Also, each cluster contained $\frac{n}{k}$ instances.
        \item $r$: Ratio of outliers of the dataset. This means that $\frac{r}{100} \ast n$ additional instances were added uniformly.
    \end{itemize}
    \item Each of the input parameters was divided into three categories: \textit{Small}, \textit{Medium} and \textit{Large}.
    \item The values for each category and input parameter for the offline phase can bee seen in \cref{tab:parametersOfflinePhase}.
    
% Please add the following required packages to your document preamble:
% \usepackage{booktabs}
\begin{table}[]
\centering
\caption{The values of the input parameters for the generation of the training datasets divided into \textit{Small}, \textit{Medium} and \textit{Large}.}
\label{tab:parametersOfflinePhase}
\begin{tabular}{@{}llll@{}}
\toprule
Parameter    & Small & Medium & Large  \\ \midrule
$n$ & 1,000 & 5,000  & 10,000 \\
$d$ & 10    & 30     & 50     \\
$k$ & 5     & 50     & 100    \\
$r$ & 0     & 33     & 66     \\ \bottomrule
\end{tabular}
\end{table}
\item For each combination of parameters one dataset was created.
\item Since there are $3^4 = 81$ combinations, the same amount of datasets was created for the training phase.
\item Because 9 metrics, 4 optimizers and 81 datasets were used, 2,916 optimizer runs were performed.
\item Each optimizer had 50 iterations, which results in 145,800 KMeans and metric executions.
\item Therefore, the range for the possible $k$ values that were sampled from the optimizers were limited to $R=[2, 2 * k_{max}]$, where $k_{max}$ denotes the highest actual k value that was used in the generated datasets.
Since $k_{max}=100$ the range was set to $R=[2, 200]$.
\item Theoretically, the range could be $[2, n]$ but this would be computational too expensive.
\item A value close to $n$ would mean that the runtime complexity of KMeans would be quadratic.
\item 
\end{itemize}
\section{Online}



% Please add the following required packages to your document preamble:
% \usepackage{booktabs}
\begin{table}[]
\centering
\caption{The values of the input parameters for the generation of the training datasets divided into \textbackslash{}textit\{Small\}, \textbackslash{}textit\{Medium\} and \textbackslash{}textit\{Large\}.}
\label{tab:parametersOfflinePhase}
\begin{tabular}{@{}lllll@{}}
\toprule
       & $n$    & $d$ & $k$ & $r$ \\ \midrule
Small  & 1,000  & 10  & 5   & 0   \\
Medium & 5,000  & 30  & 50  & 33  \\
Large  & 10,000 & 50  & 100 & 66  \\ \bottomrule
\end{tabular}
\end{table}
\subsection{Coldstart}

\subsection{Warmstart}